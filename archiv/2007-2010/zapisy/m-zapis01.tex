\documentclass[a4paper,12pt,notitlepage,oneside]{article}

\usepackage[utf8]{inputenc}
\usepackage[czech]{babel}
\usepackage[T1]{fontenc}
\usepackage{a4wide}
\usepackage[pdftex]{graphicx}
\usepackage{times}
\usepackage{array}
\usepackage{cslatexquotes}

\title{\textbf{Zápis z 1. mimořádného zasedání AS FEL}\\konaného dne 4. 5. 2007 od 9:00 hodin\\v místnosti č. 80 v dejvické budově FEL}
\author{}\date{} 

\newcommand{\hlasovani}[3]{\begin{center}
\begin{tabular}{|c|c|c|}\hline
pro & proti & zdržel se \\ \hline
#1 & #2 & #3 \\ \hline
\end{tabular}
\end{center}}

\parindent 12pt
\parskip 2pt

\begin{document}

\maketitle

\paragraph{Přítomni:} Petr BOREŠ, Mattia BUTTA, Jindřich FUKA, Petr HABALA, Karel HOFFMANN, Jan HOLUB, Josef KOLÁŘ, Vítězslav KŘÍHA, Michal KUBÍNYI, Miroslav LEV, Mirko NAVARA, Jiří NOVÁK, Petr SKALICKÝ, Jan SLÁMA
\paragraph{Omluveni:} Karol BUJAČEK, Radomír ČERNOCH, Martin DOBIÁŠ, Petr JEŽDÍK, Jaromír KAŠPAR, Michal PĚCHOUČEK, Ivan PRAVDA, Martin PŮLPITEL, Monika ŽÁKOVÁ
\paragraph{Hosté:} proděkani Hazdra, Mindl, Müller a Ripka, tajemník Mráz, ing. Blechová, ing. Dočkal

\bigskip

\textit{V. Kříha:} Zahájil zasedání a konstatoval, že bylo svoláno za účelem projednání návrhu rozpočtu na rok 2007.

\textit{I. Mráz:} Seznámil senát s materiálem Návrh rozpočtu ČVUT FEL pro rok 2007 a se souvisejícími dokumenty (např. rozpočet TPO, SVTI). Pohovořil o zdrojích financování, příjmech z hlavní činnosti, investičních a neinvestičních výdajích. Okomentoval důležité položky. Informoval např., že od letošního roku je nutné platit za využívání Betlémské kaple částku cca 3 tis. Kč za hodinu, že od příštího semestru by měla být zahájena výuka v detašovaném pracovišti FEL v Sezimově Ústí. Pohovořil o možnosti od minulého roku vytvářet Fond provozních prostředků – jde o finanční prostředky, které škola může převést do dalšího roku a použít např. na krytí případného deficitu rozpočtu  v dalších letech. V této souvislosti upozornil na předpokládaný pokles studentů v příštích letech. K položce mzdy uvedl, že ji tvoří fakultní platby, tzv. posílení a pracoviště. Pohovořil k položce náhrady za dovolenou – od letošního roku jdou za tím zdrojem, ze kterého byly mzdy v předchozím čtvrtletí vyplaceny. Vysvětlil tabulku posílení – jde o náklady na nově přijaté pracovníky servisních pracovišť. Sdělil, že na pracoviště se rozděluje 172 380 tis. Kč ve stanovených poměrech. Uvedl, že na straně investičních výdajů je volně k dispozici cca 35 mil. Představil plán investic na rok 2007. Zmínil se o rekonstrukci zasedací místnosti č. 104. Uvedl, že v následujících letech se očekává na ČVUT investiční boom, ovšem prostředky ČVUT jsou minimální, peníze by měly přijít z ministerstva školství, podílet by se měly i jednotlivé fakulty, prostavět by se mělo ročně 600 až 800 mil. Promluvil o materiálu Pravidla pro nakládání s rezervami – obsahuje např. rezervu děkana.

\bigskip

\textit{V. Kříha:} Vyzval k dotazům.

\textit{M. Kubínyi:} Otázal se na rozdělení peněz na pedagogickou činnost a výzkum – z čeho se vycházelo při stanovování poměru 80:20.
\textit{I. Mráz:} Uvedl, že jde o poměr přejatý z minulých let. Dále sdělil, že vědu chápe jako úzce spjatou s pedagogikou. Uvedl, že v 90. letech byl poměr 70:30.

\textit{M. Navara:} Otázal se, zda součástí schvalování rozpočtu jsou i související dokumenty (metodiky) nebo zda jde pouze o globální pohled.
I. Mráz, proděkan Hazdra: Odpověděli, že rozpočet má dvě fáze – rozdělení peněz z hlediska fakulty a rozdělování prostředků mezi pracoviště (metodika Kometa, VVVS). Zde je cyklus dvouletý z hlediska vytvoření stabilního prostředí.

\textit{M. Navara:} Otázal se, jakou agendu vyřizuje zaměstnanecké oddělení a
proč je na 90\% stavu z roku 1990, zda by se na něm nedalo ušetřit a
prostředky přesunout oddělení pro vědu a výzkum.

\textit{I. Mráz:} Odůvodnil zachování stavu z roku 1990. Informoval, že v současné době jsou mzdy některých pracovníků hrazeny až z pěti zdrojů. Pracovnice oddělení dohlížejí na čerpání finančních prostředků přidělených na mzdy, které musejí být vyčerpány do 100\%. Zároveň uvedl, že změny u jednotlivých pracovníků jsou prováděny měsíčně.

\textit{M. Navara:} Otázal se na rozdělení kompetencí mezi zaměstnaneckým oddělením, ekonomickým oddělením a finanční účtárnou.

\textit{I. Mráz:} Zaměstnanecké odd. – mzdy, pracovně právní vztahy, ekonomické odd. – správa investičních a neinvestičních prostředků, finanční účtárna – účetnictví.

\textit{M. Navara:} Otázal se, co bude získáno na oplátku za navýšení rozpočtu informačního systému o 25\%. Upozornil, že KOS nemá oficiální podporu – není nikde uvedena kontaktní osoba. Navrhl pozastavit financování do doby vyřešení problému. Otázal se, na koho se má na fakultě obracet v případě problémů.

\textit{I. Mráz:} Informoval, že provoz KOSu jde přes Výpočetní a informační centrum, které vychází z potřeb FEL. Za sebe uvedl, že se v případě problémů vždy obrací na proděkana Müllera.

\textit{Proděkan Mindl:} Uvedl, že v případě problémů se obrací také na proděkana Müllera, který má vazby na firmu Trill, která KOS vymyslela a spravuje.

\textit{M. Navara:} Požádal, aby v tom případě byl proděkan Müller uveden na webu jako kontaktní osoba pro KOS.

\textit{Proděkan Ripka:} Uvedl, že metodikem KOSu je ing. Halaška, a vyjádřil přesvědčení, že je tato informace všeobecně známa.

\textit{J. Fuka:} Vyjádřil názor, že jméno ing. Halašky by mělo být na webu uvedeno, neboť proděkan Müller by neměl být zatěžován těmito problémy.

\textit{M. Navara:} Konstatoval, že ani ing. Halaška není nikde uveden.

\textit{M. Lev:} Otázal se, zda jsou v tab. na s. 5 (rozdělení procent) zahrnuty údaje z tab. posílení. (Ne.)

\textit{J. Fuka:} Požádal o informaci ohledně rekonstrukce místnosti č. 104 za 2 mil. Kč. (viz. níže.)
.
\textit{P. Skalický:} V souvislosti se schvalováním rozpočtu se otázal na změny v Kometě.

\textit{Proděkan Mindl:} Uvedl, že je připraven návrh, který však počítá pouze s úpravou vlivu ankety.

\textit{P. Habala:} Upozornil, že Kometa byla schválena jen pro rok 2006, v roce 2007 již není platná a je třeba tedy schválit novou.

\textit{P. Skalický:} Vznesl dotaz na tabulku rozdělování mezd na jednotlivá pracoviště – proč jde poměr za poslední roky u kateder dolů, zatímco u ostatních pracovišť nahoru? Otázal se v souvislosti s počtem pracovníků na děkanátě (uvedeno srovnání s rokem 1990), zda existuje poměr mezi počtem pracovníků a částkou na mzdy.

\textit{I. Mráz:} V odpovědi na druhou otázku uvedl, že jde o trend daný již předchozím vedením. Co se procentuálních trendů týče, uvedl, že jde o vývoj, který byl vždy minulým vedením projednán a schválen.

\textit{J. Fuka:} Požádal o doplnění počtu pracovníků na děkanátě o sloupec s rokem 2000.

\textit{M. Lev:} Vznesl dotaz na zajištění servisu výtahů.

\textit{D. Blechová:} Informovala, že na FEL je 9 výtahů, které podléhají pravidelným kontrolám, přičemž částka 250 tis. Kč za servis zahrnuje i běžné náhradní díly, pokud nejde o generální opravy. Částka nezahrnuje energie. Uvedla, že servis zajišťuje firma Liftservis.

\textit{M. Lev:} Informoval o své zkušenosti: Za 7 výtahů platí cca 85 tisíc ročně včetně energií. Upozornil, že páternoster není výtah, ale strojní zařízení, vztahují se tedy na něj jiné předpisy. Vyzval ing. Blechovou k jednání s firmou Liftservis o snížení ceny.

\textit{D. Blechová:} Přislíbila, že bude s firmou jednat.

\textit{J. Fuka:} Připomněl ing. Blechové svůj požadavek, aby v souvislosti s budováním výtahu na Karlově náměstí byly posunuty katry oddělující centrální chodbu od bočních chodeb.

\textit{D. Blechová:} Uvedla, že o dveřích je jednáno s projektantem, jednání však není jednoduché. Informovala, že je v přípravě projekt kompletního protipožárního zabezpečení budovy E na Karlově náměstí, které bude základem. Dále konstatovala, že na to stejně nejsou peníze.

\textit{Proděkan Mindl:} Přislíbil, že jakmile se na Karlově nám. dospěje k řešení požárních úseků, budou s tím řešeny i katry.

\textit{M. Lev:} Otázal se, kdo na fakultě zajišťuje úklid.

\textit{D. Blechová:} Informovala, že úklid je zajišťován napůl soukromou firmou, napůl zaměstnanci fakulty.

\textit{P. Boreš:} Otázal se, co je lepší z hlediska ekonomické výnosnosti.

\textit{D. Blechová:} Uvedla, že fakultní zaměstnankyně vycházely levněji, ale jen bez započítání dalších nákladů, pro FEL je to navíc složitější, např. když je uklízečka nemocná.

\textit{P. Skalický:} Otázal se, z jakého důvodu jsou v rozpočtu SVTI materiály (konkrétně tonnery) zmíněny dvakrát.

\textit{M. Dočkal:} Uvedl, že první položka zahrnuje materiál, který SVTI poskytuje v rámci servisu na fakultě, druhá položka pak zahrnuje vlastní spotřebu v rámci SVTI.

\textit{J. Fuka:} Požádal o informaci o úpravě místnosti číslo 104.

\textit{M. Dočkal:} Informoval o plánované rekonstrukci. Uvedl, že vybavení audiovizuální technikou bude vyžadovat kompletní rekonstrukci místnosti počínaje elektroinstalací přes interiérové vybavení, osvětlení, klimatizaci apod. S politováním uvedl, že termín odevzdání projektu je až v pondělí 7. května. Představil pouze hrubou vizualizaci. Uvedl, že vybavení místnosti bude následovat doporučení CESNETu k vybavení zasedacích a manažerských místností.

\textit{P. Boreš:} Otázal se, kdo bude garantovat rozumné využití místnosti.

\textit{I. Mráz:} Odpověděl, že za to bude odpovídat děkan.

\textit{M. Kubínyi:} Otázal se, kolik stála rekonstrukce místnosti č. 80.

\textit{M. Dočkal:} Uvedl, že něco kolem 4 miliónů plus dodatečné náklady.

\textit{J. Fuka:} Otázal se, kolik míst bude nová zasedačka mít.

\textit{M. Dočkal:} Odpověděl, že něco kolem 20 míst.

\textit{J. Fuka:} Otázal se na způsob řešení nahrávání v místnosti 104.

\textit{M. Dočkal:} Uvedl, že je počítáno s propojením s technickým zázemím posluchárny.

\textit{Proděkan Hazdra:} Otázal se, zda bude místnost klimatizovaná. Dále vznesl dotaz, zda bude rekonstrukce synchronizována s opravou vnějšího pláště plánovanou na rok 2008.

\textit{M. Dočkal:} Uvedl, že projekt s klimatizací počítá. Konstatoval, že problém s klimatizací je v celé budově. Informoval, že místnost bude dopředu připravena na napojení klimatizace po zklimatizování celého patra. Na druhou otázku odpověděl, že rekonstrukce není synchronizována s úpravou fasády.

\textit{P. Boreš:} Konstatoval, že při výměně vnějšího pláště se vyměňují okna, takže se v místnosti budou vyměňovat stěny po stranách místnosti. Vyjádřil obavy z negativních dopadů na technické vybavení.

\textit{M. Dočkal:} Uvedl, že se s těmito komplikacemi počítá a jsou zvládnutelné.

\textit{J. Fuka:} Vyjádřil pochyby – výměna pláště bude mít negativní dopad na podlahovou krytinu. Vznesl otázku, zda je tak urgentní provést rekonstrukci již nyní.

\textit{I. Mráz:} Upozornil, že opláštění nemusí proběhnout za rok. Připomněl, že fasády nejsou v kompetenci fakulty. Vyjádřil obavu, že investice budou utlumeny tím, že na ně nebude. Proděkan Mindl: Konstatoval, že podle generelu by v příštích dvou letech měla být realizována rekonstrukce opláštění budovy FEL, ovšem v závislosti na penězích.

\textit{J. Fuka:} Shrnul, že pokud existuje názor, že opláštění nebude v roce 2008, nemá smysl čekat s rekonstrukcí místnosti.

\textit{Proděkan Mindl:} Uvedl, že z hlediska krácení finančních prostředků se budou dělat investice na vybrané akce – opláštění bude mít priority.

\textit{J. Sláma:} Otázal se, odkud začne rekonstrukce opláštění.

\textit{Proděkan Mindl:} Informoval, že se bude pokračovat od strojní fakulty.
\textit{J. Sláma:} Zmínil možnost změny pořadí opravy fasád.

\textit{Proděkan Mindl:} Není schopen říci, zda je postup technicky řešitelný. Je o tom možno v rámci přípravy akce uvažovat.

\textit{M. Kubínyi:} Otázal se na možnost výměny oken místnosti dopředu.

\textit{V. Kříha:} Upřesnil, že se vyměňuje kompletně celý plášť a ztlušťuje se.

\textit{Proděkan Mindl:} Zdůraznil, že investorem akce je rektorát.

\textit{P. Boreš:} Otázal se, kolik stojí umístění dataprojektoru, plátna a přípojky pro notebook ve standardní učebně.

\textit{M. Dočkal:} Uvedl, že přibližně 80 tisíc.

\textit{P. Skalický:} Upozornil na nejednotné vybavení jednotlivých učeben didaktickou technikou (někam je třeba brát si notebook, jinam stačí flashka). Otázal se, zda se počítá s jednotným přístupem.

\textit{M. Dočkal:} Pohovořil o historii vybavování učeben didaktickou technikou. Uvedl, že nyní přibývá vyučujících, kteří raději přijdou se svými notebooky, proto se v učebnách střední velikosti dělá jen dataprojektor, připojení na notebook a počítačová síť.

\textit{J. Novák:} Vznesl dotaz na položku posílení, kde jsou uvedeny převody z minulých let. Otázal se, zda by nebylo lepší položky, kde posílení nastalo již v minulých letech, zahrnout pod rozpočet příslušné organizační složky fakulty.

\textit{I. Mráz:} Uvedl, že při sestavování rozpočtu navazoval na práci svého předchůdce. Souhlasil s tím, že by to bylo lepší. Navrhl pro tento rok ponechat rozpočet tak, jak je, pro příští se případně dohodnout, že posílení by bylo zapracováno do rozdělovacích procent.

\textit{J. Fuka:} Navrhl na podzim v senátu předjednat nové procentuální rozdělení zahrnující změny, aby to bylo na příští rozpočet připraveno.

\textit{J. Novák:} Vyjádřil názor, že by posílení mělo být zahrnuto do procent, aby bylo jasné, jak to ve skutečnosti je.

\textit{I. Mráz:} Za sebe uvedl, že není problém čísla posléze zahrnout do procent. Upozornil však, že nejsou přítomni všichni členové senátu, kteří by později mohli žádat znovu vysvětlení. Odůvodnil tabulku - takto je vidět, co tvoří navýšení.

\textit{J. Fuka:} Pro tento rozpočet by to ponechal, neboť to ukazuje změny.

\textit{P. Skalický:} Otázal se, zda se procenta mění v závislosti na snížení/zvýšení pracovníků kateder. I. Mráz: Upozornil na výkonové ukazatele. Dále upozornil, že katedry jsou financovány z více zdrojů.

\textit{J. Kolář:} Tabulku posílení by zahrnul rovnou do procent.

\textit{M. Navara:} Vyjádřil nespokojenost, ale konstatoval, že je lepší špatný rozpočet než žádný. Vyslovil se pro ponechání rozpočtu v předložené podobě, zároveň však pro to, aby příští rozpočet vycházel z jiných principů.

\textit{I. Mráz:} Zmínil možnost započítání do procent a ponechání tabulky posílení pouze pro vysvětelní.

\textit{P. Boreš:} Vyjádřil podporu návrhu I. Mráze. Ohledně fakultních plateb vyjádřil následující: k náhradě dovolené navrhl, aby senát dal doporučení, že by se měl zpracovat přehled čerpání po katedrách a v případě nerovnoměrného čerpání zahrnout do výkazu odměn. U výuky samoplátců, získání studenta CŽV navrhl stanovit jasná pravidla a vypracování zpětného zhodnocení.

\textit{P. Habala:} Navrhl, aby senát dal najevo neformálním hlasováním, zda by se mělo posílení zahrnout do normálního rozpisu mezd – a to buď již tento rok, nebo až příští rok – nebo ponechat tabulku jen pro úplně nové posílení.

\newpage\paragraph{Neformální hlasování:}
\begin{itemize}
\item Pro provedení změn již v rozpočtu na letošní rok se vyslovilo 6 senátorů.
\item Pro provedení změn v rozpočtu na příští rok se vyslovilo 5 senátorů.
\item Pro to, aby byly v položce posílení zachovány pouze čerstvé změny, se vyslovilo 11 senátorů. 
\end{itemize}
\textbf{Senát se vyslovil, aby v tabulce byly ponechány jen čerstvé změny.}

\bigskip\bigskip

\textit{I. Mráz:} Konstatoval, že návrh rozpočtu předkládá děkan, z dnešního jednání může vzejít doporučení, jak by měl rozpočet vypadat. Vyjádřil názor, že po oficiálním schválení senátem by se zanesl sloupec – po zahrnutí posílení.

\textit{J. Fuka:} Ponechal by změny až na příští rozpočet.

\textit{I. Mráz:} Konstatoval, že je třeba vymyslet, jakým vhodným způsobem prezentovat změny, k nimž došlo.

\textit{J. Fuka:} Otázal se, co se stalo s loňskou rezervou ve výši cca 11 mil.

\textit{I. Mráz:} Odpověděl, že součástí rozpočtu nejsou fondy, stavy fondů a převodů budou ve Výroční zprávě o hospodaření.

\textit{J. Fuka:} Odůvodnil svůj dotaz – chybí mu představa, kolik má fakultave fondu provozních prostředků peněz. (8. mil. Kč)

\textit{I. Mráz:} Přislíbil, že mimo rozpočet dodá údaje, jak vypadají fondy FEL.

\textit{J. Kolář:} Zeptal se, zda se jedná o rezervu určenou pro budoucnost, nebo zda bude čerpána již brzy v případě propadu počtu studentů.

\textit{I. Mráz:} Uvedl, že rezerva je určena pro případ dramatického poklesu počtu studentů na FEL, pokles na jednotlivých katedrách hlídají meze v KOMETĚ.

\textit{Proděkan Ripka:} Upozornil, že pracovníci kateder jsou zaměstnanci fakulty, která musí zasáhnout (z rezervy) v případě, kdy na ně katedra nemá peníze. Zmínil problémy na katedře fyziky. Dále sdělil, že problém může nastat už letos - pokud se sníží počet studentů EaI z 1200 na 800, bude to mít velký dopad na některé katedry. Zmínil také, že pokud katedra narazí na 10\% zarážku, začnou ji dotovat ostatní katedry. Uvedl, že pro tyto případy podporuje položku 2 miliony rezervy v děkanově pravomoci.

\textit{J. Kolář:} Otázal se, zda by katedrám nepomohlo zvýšení vlivu ankety.

\textit{M. Navara:} Požádal o informaci, nakolik se hodnocení z ankety promítlo v minulém období do rozdělení peněz.

\textit{Proděkan Mindl:} Pohovořil o vlivu ankety. Z důvodu zachování stabilního prostředí požádal o schválení stejného koeficientu jako loni.

\textit{Proděkan Müller:} Ukázal čísla – dopad ankety na finanční prostředky kateder za letní semestr s koeficientem 0,5 a 1. Shrnul, že některé katedry zvýšením koeficientu mírně získají, jiné mírně ztratí.

\textit{M. Navara:} Vyjádřil překvapení, jak malý to má vliv.

\textit{P. Habala:} Uvedl, že metodika bude potřebovat výraznější zásahy a analýzu – upozornil, že každá katedra dostává za započitatelnou hodinu jinou hodnotu. Ponechal by metodiku zatím víceméně stejnou, aby byla získána další data. Zároveň vyjádřil předpoklad, že nové vedení bude chtít metodiku změnit.

\textit{Proděkan Mindl:} Uvedl, že rozdíly v započitatelných hodinách vzniknou v okamžiku, kdy se začne dělat klouzavý průměr. Informoval o uplatňování metodiky Kometa v případě otevření výuky na detašovaném pracovišti v Sezimově Ústí – návrh metodiky doplněn o větu na s. 5: „V případě výuky na detašovaném pracovišti mimo Prahu se aplikují výše uvedená kritéria samostatně pro každé pracoviště.“

\textit{P. Boreš:} Přimluvil se k ponechání současného stavu. Upozornil, že ve výsledcích ankety není zveřejňováno výsledné číslo, které se pak používá pro metodiku.

\textit{Proděkan Mindl:} Pohovořil o způsobu výpočtu známky.

\textit{J. Sláma:} Vznesl dotaz na položku AS FEL (80 tis.).

\textit{I. Mráz:} Odpověděl, že jde o novou položku, do které budou spadat náklady spojené s činností AS FEL: např. zápis, případné tlumočení či vizitky pro senátory, školení, výjezdní zasedání.

\textit{J. Kolář:} Vznesl dotaz náhrady za dovolenou – jsou na dvou místech.

\textit{I. Mráz:} Vysvětlil systém vyplácení dovolených – dle průměrné mzdy za předchozí čtvrtletí. Uvedl, že od letošního roku náhrady za dovolenou jsou vypláceny podle toho, z jakého zdroje byly vypláceny mzdy v minulém čtvrtletí – z čeho šly mzdy, jdou i dovolené.

\textit{K. Hoffmann:} Otázal se na odměny z grantů a záměrů.

\textit{I. Mráz:} Uvedl, že ty jdou mimo rozpočet. Upozornil na problém, kdy grant skončí, je vyplacena odměna, pak si pracovník vezme dovolenou a grant už nedodá peníze na náhradu za dovolenou, to se pak dotuje z kmenových prostředků.

\textit{J. Fuka:} Zeptal se, zda jdou odměny stanovit jako "za čtvrtletí" a "za rok", pak by se to rozpočítalo jednou čtvrtinou.

\textit{I. Mráz:} Uvedl, že je doporučení odměny vyplácet jako jednorázové ne jako roční, protože roční odměna zatíží rozpočet celého roku. Shrnul, že efekt by byl v podstatě stejný.

\textit{J. Fuka:} Nárůst rezervy děkana na neinvestiční náklady a mzdy (z 1 mil. na 2 mil.) se mu zdá příliš vysoký. Navrhl 1,5 mil.

\textit{P. Boreš:} Větší výši považuje za rozumnou.

\textit{J. Fuka:} Za ekonomickou komisi vyslovil požadavek na spočítání administrativních pracovníků na FEL celkově (děkanát i katedry) a zpracování přehledu toho, jaké činnosti vykonávají.

\textit{I. Mráz:} Uvedl, že je třeba požádat vedoucí kateder o informace. Informoval, že v rámci 7. rámcového programu se bude počítat režie včetně administrativy na FEL a na rektorátě. Dále informoval, že na rok 2007 byla původně plánována i klimatizace 2 pater v jižním křídle za 1,6 mil. korun, rekonstrukci místnosti č. 104 však považoval za přednější.

\textit{J. Fuka:} Vyzval ke schválení rekonstrukce, konečné rozhodnutí by však ponechal na vedení.

\textit{P. Skalický:} Uvedl, že v případě, že se nebude provádět rekonstrukce místnosti 104, peníze by mohly jít do rezervy. V souvislosti s diskusí o klimatizaci sdělil, že mohou pomoci i vnější žaluzie.

\textit{I. Mráz:} Vyjádřil názor, že v zasedací místnosti a u pana děkana by klimatizace měla určitě být.

\textit{K. Hoffmann:} Otázal se, zda se výměna pláště projeví v poklesu nákladů na spotřebu energií.

\textit{I. Mráz:} Odpověděl, že investice do oken se už vrátila. Vyjádřil názor, že opláštění nebude znamenat snížené výdaje na energie, neboť ceny rostou, výdaje by ale neměly vzrůst.


\bigskip\bigskip\bigskip\bigskip\bigskip\bigskip


V Praze dne


\bigskip\bigskip\bigskip

\begin{center}
\begin{tabular}{p{4cm}p{4cm}p{4cm}}
V. Kříha & Michal Pěchouček & Petr Habala \\
předseda AS FEL & předsedající AS FEL & tajemník AS FEL
\end{tabular}
\end{center}


\end{document}