\documentclass[a4paper,12pt,notitlepage,oneside]{article}

\usepackage[utf8]{inputenc}
\usepackage[czech]{babel}
\usepackage[T1]{fontenc}
\usepackage{a4wide}
\usepackage[pdftex]{graphicx}
\usepackage{times}
\usepackage{array}
\usepackage{cslatexquotes}

\title{\textbf{Zápis z~3. řádného zasedání AS FEL}\\konaného dne 13. 4. 2007 od 9.00 hodin\\v místnosti č. 80 v~dejvické budově FEL}
\author{}\date{} 

\newcommand{\hlasovani}[3]{\begin{center}
\begin{tabular}{|c|c|c|}\hline
pro& proti & zdržel se \\ \hline
#1 & #2 & #3 \\ \hline
\end{tabular}
\end{center}}

\parindent 12pt
\parskip 2pt

\begin{document}

\maketitle

\paragraph{Přítomni:} Petr BOREŠ, Karol BUJAČEK, Mattia BUTTA, Radomír ČERNOCH, Martin DOBIÁŠ, Jindřich FUKA, Petr HABALA, Karel HOFFMANN, Jan HOLUB, Josef KOLÁŘ, Vítězslav KŘÍHA, Miroslav LEV, Mirko NAVARA, Jiří NOVÁK, Michal PĚCHOUČEK, Ivan PRAVDA, Martin PŮLPITEL, Petr SKALICKÝ, Jan SLÁMA, Monika ŽÁKOVÁ
\paragraph{Omluveni:} Petr Ježdík, Jaromír Kašpar, Michal Kubínyi

\paragraph{Hosté:} P. Hazdra, P. Ripka, prof. Škvor



%%%%%%%%%%%%%%%%%%%%%%%%%%%%%%%%%%%%%%%%%%%%%%%%%%%%%%%%%%%%%%%%%%%%%%%%%%%%%%%%
\section{Schválení programu zasedání}
\textit{M. Butta:} Požádal s~ohledem na možnost tlumočení senátory o~pomalejší mluvu.

\paragraph{Hlasování o~schválení programu zasedání:}
\hlasovani{20}{0}{0}

\paragraph{Program 3. zasedání byl schválen v~této podobě:}

\renewcommand{\labelenumii}{\alph{enumii})}
\newcounter{mycount}
\begin{enumerate}
\item Schválení programu zasedání
\item Schválení zápisů z~minulých zasedání
\item Volba předsedajícího a tajemníka
\item Projednání návrhů děkana:
	\begin{enumerate}
	\item Návrh na zrušení katedry TV
	\item Návrh na jmenování prof. Ing. Vladimíra Kučery, DrSc., Dr.h.c. členem vědecké rady fakulty
	\end{enumerate}
\item Příprava volby děkana
\item Různé
	\begin{enumerate}
	\item Semináře ACSA
	\item Schválení seznamu členů konferencí
	\end{enumerate}
\item Datum příštího zasedání
\end{enumerate}



%%%%%%%%%%%%%%%%%%%%%%%%%%%%%%%%%%%%%%%%%%%%%%%%%%%%%%%%%%%%%%%%%%%%%%%%%%%%%%%%
\section{Schválení zápisů z~minulých zasedání}

\textit{M. Navara:} Požádal o~schválení zápisů na příštím zasedání s~ohledem na chybějící údaje.

\paragraph{Hlasování o~tom, zda mají být úpravy provedeny ihned:}
\hlasovani{18}{2}{0}
\textbf{Zápisy budou projednávány na tomto zasedání.}

\vskip 12pt
\noindent Promítány zápisy a diskutovány drobné úpravy.

\paragraph{Hlasování o~schválení zápisu z~1. zasedání:}
\hlasovani{16}{0}{4}
\textbf{Zápis byl schválen.}

\paragraph{Hlasování o~schválení zápisu z~2. zasedání:}
\hlasovani{18}{0}{1}
\textbf{Zápis byl schválen.}


%%%%%%%%%%%%%%%%%%%%%%%%%%%%%%%%%%%%%%%%%%%%%%%%%%%%%%%%%%%%%%%%%%%%%%%%%%%%%%%%
\newpage\section{Volba předsedajícího a tajemníka}
\textit{M. Pěchouček:} Navrhl sebe jako předsedajícího a P. Habalu jako tajemníka na další tři zasedání.

\paragraph{Hlasování o~složení nového předsednictva: tajemník M. Pěchouček, předsedající P. Habala:}
\hlasovani{17}{0}{2}
\textbf{Předsednictvo v~uvedeném složení bylo schváleno.}


%%%%%%%%%%%%%%%%%%%%%%%%%%%%%%%%%%%%%%%%%%%%%%%%%%%%%%%%%%%%%%%%%%%%%%%%%%%%%%%%
\section{Projednání návrhů děkana}
\subsection{Návrh na zrušení katedry TV}
\textit{P. Boreš:} Konstatoval, že byla otevřena diskuse na fóru a že mezi studenty se skutečně hovoří o~tom, že se za TV bude platit, neví se však, z~jakého zdroje informace pochází. Konstatoval, že není k~dispozici zpráva od Mgr. Valjenta, bývalého vedoucího katedry TV. Vznesl dotaz k~vedení, kdo řekne senátu, jak to bude vypadat s~TV v~příštím roce.

\textit{Z. Škvor:} Navrhl dotázat se ředitele ústavu doc. Trenka.

\textit{Proděkan Ripka:} Vyjádřil názor, že není možné, aby se platilo za každý TV. Uvedl, že, pokud by tomu tak však bylo, vedení FEL poruší slib, který dalo, a zruší povinný TV. Připomněl, že např. za bowling se platilo i na FEL. Dále pohovořil, že ke zrušení katedry je potřeba změna statutu. Informoval, že statut je připraven, vedení však rozhodlo, že jej předloží nový děkan.

\textit{P. Boreš:} Konstatoval, že bod musí být odložen do schválení nového statutu FEL.

\paragraph{Hlasování o~odložení bodu:}
\hlasovani{20}{0}{0}
\textbf{Bod byl odložen.}

\subsection{Návrh na jmenování prof. Ing. Vladimíra Kučery, DrSc., Dr.h.c. členem vědecké rady fakulty}
\textit{V. Kříha:} Jako předseda AS FEL informoval o~obdržení návrhu na jmenování a předložil jej ke schválení. Zdůraznil, že návrh byl předložen prof. Škvorem ještě v~době, kdy byl děkanem FEL.

\textit{M. Navara:} Podpořil jmenování a vyjádřil přesvědčení, že prof. Kučera do VR patří.

\textit{P. Boreš:} Souhlasil s~M. Navarou, návrh však označil za nesystémový, neboť nově zvolený děkan pravděpodobně vytvoří komplexní návrh na doplnění VR. Vyslovil se pro odložení projednávání návrhu až do doby po zvolení nového děkana.

\textit{P. Habala:} Vyjádřil názor, že prof. Kučera by se měl stát členem VR  důstojnějším způsobem a že novém děkanovi by měl být ponechán prostor podívat se na VR komplexně. Vyjádřil přesvědčení, že se prof. Kučera stane členem VR na návrh nového děkana.

\textit{Z. Škvor:} Informoval, že návrh na jmenování prof. Kučery na poslední chvíli byl důsledkem slibu předsedovi předchozího senátu M. Bačovskému, že nepředloží nikoho, koho už senát jednou neschválil. Vyjádřil názor, že bývalí děkanové by neměli být automaticky vyškrtáváni z~VR.

\textit{M. Pěchouček:} Vyslovil se pro odložení bodu.

\paragraph{Hlasování o~odložení návrhu na jmenování profesora Kučery členem VR FEL:}
\hlasovani{11}{3}{6}
\textbf{Návrh byl odložen.}


%%%%%%%%%%%%%%%%%%%%%%%%%%%%%%%%%%%%%%%%%%%%%%%%%%%%%%%%%%%%%%%%%%%%%%%%%%%%%%%%
\section{Příprava volby děkana}

\textit{P. Boreš:} Představil navržené materiály s~úpravami a otevřel k~nim diskusi.

\textit{J. Fuka:} Vznesl dotaz, zda by bylo možno zkrátit funkční období děkana tak, aby bylo v~souladu s~dalšími volenými funkcemi na fakultě.

\textit{P. Boreš:} Vyjádřil názor, že zkrácení funkčního období děkana by bylo dalším provizoriem.

\textit{Z. Škvor:} Sdělil, že ze zákona je funkční období čtyřleté, mohlo by být zkráceno pouze  rozhodnutím rektora po projednání s~AS FEL a schválení AS ČVUT.

\textit{P. Boreš:} Pohovořil o~navržených termínech: registrace kandidátů do 15. 5. 2007 do 12.00 hodin, setkání se členy AS FEL 18. 5., setkání s~AO 23. 5., termín volby 25. 5. Navrhl přesunout termín volby na 1. 6. 2007 s~ohledem na termín konání setkání s~AO 23. 5. Vyjádřil obavy z~krátké doby mezi diskusí a volbou - mohou se objevit případné problémy, kdy bude potřeba ověřit slova kandidátů.

\textit{P. Habala:} Vyslovil se proti a vyjádřil předpoklad, že taková situace pravděpodobně nenastane. Navrhl v~tom případě spíše přesunout setkání s~AO na pondělí 21. 5. (Diskutováno později a ponechán původní termín.)

\textit{P. Boreš:} Vyjádřil obavy z~krátké doby na přípravu besedy s~AO.

\textit{P. Habala:} Vyslovil se, že příprava se dá udělat předem.
M. Půlpitel, J. Kolář, M. Pěchouček: Vyslovili se pro co nejrychlejší průběh voleb.

\textit{I. Pravda:} Upozornil, že se blíží doba dovolených a s~tím spojená možnost nepřítomnosti kandidátů.

\textit{Proděkan Ripka:} Upozornil, že je v~pravomoci senátu sejít se 25. 5. a případně rozhodnout o~odložení volby.

\newpage
\paragraph{Hlasování o~termínu volby:}
\begin{itemize}
\item \textbf{Varianta 1 - termín 25. 5. 2007}
\end{itemize}
\hlasovani{16}{1}{3}
\begin{itemize}
\item 
\textbf{Varianta 2 - termín 1. 6. 2007}
\end{itemize}
\hlasovani{2}{8}{10}
\textbf{Ponechán původní termín 25. 5. 2007.}

V~rámci diskuse o~formulaci textu hlavního materiálu, zejména o~formulaci požadavků na děkanský projekt:

\textit{M. Navara:} Připomněl požadavek na přehled dosavadní činnosti a kvalifikace. Vyjádřil názor, že doporučené body děkanského projektu jsou specifické pro FEL a diskriminují zájemce zvenku.

\textit{M. Pěchouček:} Souhlasil s~tím, že některé to diskvalifikuje, chce však děkana, který bude umět věci řešit.

\textit{M. Navara:} Uvedl, že i pro něj je důležité, aby uměl problémy řešit, ale až ve chvíli, kdy děkanem bude, ne ve chvíli, kdy se hlásí do konkurzu.

\textit{M. Pěchouček:}  Souhlasil s~řešením, že body budou uvedeny v~materiálu o~úroveň níže.

\textit{V. Kříha:} Vyjádřil názor, že návrhy kandidáta zvenku a kandidáta z~FEL budou posuzovány odlišným způsobem.

\textit{M. Navara:} Vyjádřil názor, že kandidát zvenku bude zvažovat své šance - body by jej mohly odradit.

\textit{P. Boreš:} Shrnul řešení: Body by neměly být uvedeny v~hlavním materiálu, kde by bylo pouze \uv{AS FEL doporučuje zaměřit se na prezentaci dosavadních výsledků a hlavních cílů ve funkci.} V~materiálu o~úroveň níže by měly být konkrétnější záležitosti.

\textit{M. Navara:} Přivítal toto řešení.
K. Hoffmann: Souhlasil s~přesunutím bodů o~úroveň níže a navrhl průběžně je doplňovat o~obsah schránky diskusního fóra "Kde nás tlačí bota".

\textit{P. Habala:} Navrhl v~této souvislosti formulaci: \uv{Členy AS budou zajímat zejména názory na tyto otázky?}

\textit{Z. Škvor:} Upozornil, že od 1. 1. 2007 je děkan veřejným funkcionářem ve smyslu zákona o~střetu zájmů, musí tedy předkládat majetková přiznání, a to nejen on, ale i jeho manželka.

\paragraph{Hlasování o~textu hlavního materiálu (vyhlášení volby), který bude vyvěšen na stránkách fakulty:}
\hlasovani{20}{0}{0}


\textbf{Text materiálu byl schválen v~podobě:}\\

\noindent{\it Akademický senát Fakulty elektrotechnické ČVUT v~Praze vyhlašuje podle § 27 odst. 1 písm. g) zákona č. 111/ Sb. v~platném znění volbu kandidáta na děkana fakulty pro čtyřleté funkční období 2007 - 2011.

\noindent\textbf{Harmonogram volby:}
\begin{center}
\begin{tabular}{|l|l|}\hline
Registrace kandidátů: & 
do 15. 5. 2007 do 12:00 hod.\\ \hline
Setkání se členy AS FEL: & 
18. 5. 2007\\ \hline
Setkání s~akademickou obcí: & 
23. 5. 2007\\ \hline
Termín volby: & 
25. 5. 2007\\ \hline
\end{tabular}
\end{center}

\noindent\textbf{Přihláška musí obsahovat:}\\
a) písemný souhlas s~kandidaturou\\
b) děkanský projekt\\
c) zkrácenou verzi děkanského projektu v~rozsahu jedné strany formátu A4


\noindent\textbf{Uvedené materiály doručte na adresu:}
\begin{center}
\begin{tabular}{|l|l|}\hline
Akademický senát Fakulty elektrotechnické Českého vysokého učení technického v~Praze\\
k~rukám předsedy MUDr. Ing. Vítězslava Kříhy, Ph.D.\\
Technická 2\\
166 27 Praha 6\\\hline
\end{tabular}
\end{center}

\noindent v~podobě papírového dokumentu s~podpisem kandidáta, materiál podle bodu b), c) navíc v~elektronické podobě (na CD), vše v~zalepené obálce výrazně označené v~levém horním rohu nápisem: "Kandidát na děkana - NEOTEVÍRAT!". Z~hlediska termínu je rozhodující doručení do podatelny FEL. Po uzavření registrace kandidátů budou zkrácené verze děkanských projektů zveřejněny na nástěnkách FEL. Děkanský projekt by měl zahrnovat dosavadní výsledky a zkušenosti kandidáta a jeho názory na důležité aspekty dalšího rozvoje FEL ČVUT. Funkce děkana fakulty je podmíněna předložením lustračního osvědčení podle zákona 451/1991 Sb. v~platném znění. Děkan je veřejným funkcionářem ve smyslu zákona o~střetu zájmů 159/2006 Sb. Uvedený termín setkání s~akademickou obcí je orientační a s~ohledem na počet kandidátů a prostorové možnosti může být upraven.}

\vskip 24pt

\textit{P. Boreš:} Seznámil senátory s~návrhem dát inzerát do Hospodářských novin a Mladé Fronty Dnes a zachovat text od minulého senátu.

\textit{Proděkan Ripka:} Doporučil doplnit informaci o~uzávěrce přihlášek.

\textit{V. Kříha:} Navrhl dát inzerát pouze do jedněch novin.
Konečné znění inzerátu do Hospodářských novin: "Akademický senát Fakulty elektrotechnické ČVUT v~Praze vyhlašuje volby kandidáta na děkana. Uzávěrka přihlášek dne 15. 5. 2007. Bližší informace naleznete na www.fel.cvut.cz." 

V~rámci diskuse o~textu materiálu o~úroveň níže a o~průběhu setkání kandidátů s~AO:

\textit{P. Boreš:} Informoval o~návrhu na průběh setkání: Kandidáti postupně představí svůj program v~krátkém projevu, poté bude následovat diskuse se členy senátu.

\textit{P. Habala:} Informoval, že při minulé volbě beseda neprobíhala hromadně, kandidáti přicházeli postupně.

\textit{P. Boreš:} Uvedl, že důsledkem tohoto řešení by bylo omezení diskuse - byl by třeba časový rozvrh.

\textit{M. Pěchouček:} Vyslovil se pro to, aby kandidáti mohli hovořit se senátory individuálně bez přítomnosti svých konkurentů.

\textit{P. Boreš:} Konstatoval, že jednání senátu je veřejné a nikdo nemůže být vyhozen.

\textit{M. Pěchouček:} Navrhl kandidáty požádat, aby odešli.

\textit{Z. Škvor:} Informoval o~průběhu setkání při loňské volbě rektora: Kandidáti byli soustředěni v~místnosti bez internetu.

\textit{P. Boreš:} Konstatoval, že kandidáti tedy půjdou jednotlivě. Uvedl, že na diskusním fóru FEL budou otevřena dvě témata v~souvislosti s~volbou děkana - 1. diskuse směřující k~úvahám o~vhodných kandidátech, 2. diskuse k~přihlášeným kandidátům (otevřena po uzávěrce přihlášek).

\textbf{Z~diskuse o~formulaci \uv{doporučených bodů} děkanského projektu vyplynul text následujícího odstavce (materiál zůstává otevřený):}

{\it Členy AS zajímají názory uchazeče zejména na následující témata: koexistence pedagogické činnosti, vědecké činnosti a spolupráce s~průmyslem na FEL, řízení kvality a odměňování pedagogické a vědecké činnosti na FEL, budoucnost studijních programů na FEL, koexistence elektrotechniky a informatiky, výběrovost studia, uvažovaný tým spolupracovníků (obecnější pojetí, ne nutně seznam proděkanů).}

Diskutován okamžik otevírání obálek - úterý 15. 5. ve 12.15 hodin.
\textit{P. Boreš:} Konstatoval, že materiály budou vyvěšeny v~průběhu středy či čtvrtka.

\paragraph{Hlasování o~schválení znění materiálů týkajících se voleb děkana:}
\hlasovani{18}{0}{0}
\textbf{Materiály byly schváleny.}


%%%%%%%%%%%%%%%%%%%%%%%%%%%%%%%%%%%%%%%%%%%%%%%%%%%%%%%%%%%%%%%%%%%%%%%%%%%%%%%%
\section{Různé}
\subsection{Semináře ACSA}
\textit{M. Půlpitel:} Informoval o~konání bezplatných seminářů u~ACSA v~Brně: Akademické senáty (4.5.), Komunikace a argumentace (27.4.). Informoval, že se mu jako zájemci přihlásili R. Černoch, K. Bujaček. Podotkl, že v~minulosti se již bývalí senátoři těchto seminářů účastnili a ohlasy byly kladné.

\textit{V. Kříha:} Informoval, že konání semináře Komunikace a argumentace je plánováno i v~Praze na podzim.
 
\paragraph{Hlasování o~usnesení: AS FEL vysílá studenty K. Bujačka, R. Černocha, M. Půlpitla na semináře ACSA v~Brně (27.4. a 4. 5. 2007):}
\hlasovani{13}{0}{5}
\textbf{Usnesení bylo schváleno.}

\subsection{Schválení seznamu členů konferencí}
\textit{M. Pěchouček:} Představil seznam členů "diskusní skupiny" a "e-mailové konference". Ke skupině "přátelé AS FEL" uvedl, že ji tvoří senátoři velkého senátu, někteří z~vedení, paní Vlčková, rektor a pan emeritní děkan. Navrhl, chce-li senát zachovat tuto skupinu, aby seznam členů byl zveřejněn nebo aby se explicitně vyjmenovaly funkce, které jsou ve skupině zastoupeny.

\textit{P. Boreš:} Informoval o~návrhu nechat zavést skupinu \uv{vedení FEL} a přidělit jí přístup do sekce \uv{AS FEL a přátelé}.

\textit{M. Pěchouček:} Odůvodnil návrh tím, že vedení nemá možnost diskutovat se senátory na fóru. Za klíčovou otázku označil skupinu "přátelé" - buď ji zanechat, jak je, tedy že senát bude schvalovat každého člena skupiny, nebo nahradit sekcí "členové AS ČVUT".

\textit{M. Navara:} Vznesl námitky na nepřehlednost připraveného materiálu.

\textit{P. Habala:} Navrhl přejmenování skupiny "AS FEL a přátelé" na "Pracovní skupina AS FEL".

\textit{M. Pěchouček:} Vyjádřil názor, že by se mělo zařazovat do seznamů na základě funkcí.

\textit{P. Boreš:} Připomněl, že diskuse kolem skupiny "AS FEL a přátelé" vznikla tím, že lidé, kteří nebyli v~hierarchii funkcí, vznesli připomínku, že chtějí být zařazeni do diskuse.

\textit{M. Pěchouček:} Vyjádřil obavy z~posuzování jednotlivců - kdo má a kdo nemá být členem skupiny.

\textit{Z. Škvor:} Upozornil, že v~seznamech jsou i lidé, kteří již nejsou v~právním vztahu s~FEL.

\textit{M. Pěchouček:} Uvedl, že řešením by bylo po nějaké době zrušit pracovní skupinu \uv{bývalý senát}.

\textit{P. Habala:} Upozornil, že zavede-li senát zařazování dle funkcí, odmítá dopředu lidi, kteří funkci nemají. Nevidí rozdíl mezi oběma způsoby zařazování.

\textit{P. Boreš:} Uvedl, že je třeba zaujmout stanovisko ke skupině "přátelé AS FEL".

\textit{M. Navara:} Navrhl odložení bodu na příští zasedání z~důvodů nedostatečné připravenosti. Vznesl požadavek na hierarchickou strukturu materiálu.

\textit{M. Pěchouček:} Ohradil se proti tomu, že je materiál nedostatečně připravený.

\textit{J. Fuka:} Otázal se na skupinu \uv{bývalý senát}.

\textit{P. Habala:} Uvedl, že nejde o~skupinu na diskusním fóru. Vyzval k~hlasování o~tom, zda senát chce zařazovat do skupiny na základě funkcí nebo podle jmen.

\newpage\paragraph{Hlasování:}
\begin{itemize}
\item \textbf{Varianta 1 - zařazování dle funkcí (senát, minulý senát, vedení):}
\end{itemize}
\hlasovani{2}{14}{3}

\begin{itemize}
\item 
\textbf{Varianta 2 - zařazování dle jmen (senát a vedení implicitně, ostatní na požádání):}
\end{itemize}
\hlasovani{16}{1}{2}
\textbf{Senát se vyslovil pro zařazování do diskusní skupiny podle jmen.}

\textit{M. Pěchouček:} Stáhl svůj návrh na zavedení sekce \uv{AS FEL a vedení}.

\textit{P. Boreš:} Konstatoval, že je třeba do příštího zasedání sestavit seznam členů skupiny "Pracovní skupina FEL", který musí být zveřejněn. Uvedl, že ti, kdo chtějí být zařazeni, by měli projevit zájem

\textit{P. Habala:} Informoval, že již má seznam lidí, kteří projevili zájem být v~e-mailové konferenci senátu, a přislíbil, že shromáždí i zájemce o~členství ve skupině "Pracovní skupina FEL".

\textit{M. Pěchouček:} Navrhl e-mailovou konferenci využívat jako primární metodu komunikace pro závazné informace a rozdělit ji do třech skupin \uv{senát only}, diskusní skupinu \uv{senát a vedení} a skupinu \uv{senát extended} (senát a senátoři AS ČVUT).

\textit{J. Fuka:} Souhlasil s~navrženým řešením.

\textit{P. Boreš:} Varoval před množstvím e-mailů a posíláním příloh.

\textit{M. Pěchouček:} Navrhl řešení - zakázat posílání příloh a ukládat je do úložiště.

\textit{P. Habala:} Navrhl pouze dvě skupiny \uv{senát} (senát, senátoři velkého senátu, vedení, zástupci v~RVŠ, rektor, paní Vlčková) a \uv{senát only}.

\textit{J. Fuka:} Otázal se, zda budou bývalí senátoři členy nějaké e-mailové skupiny. (Ne.)

\textit{R. Černoch:} Promluvil o~technických náležitostech plánovaného úložiště.

\subsection{Informace o~defibrilátoru}
\textit{V. Kříha:} Informoval o~tom, že vedení vyslalo zástupce na seminář SCIO. Konstatoval, že FEL k~výsledkům národních srovnávacích zkoušek přihlíží. Dále informoval, že proběhlo 1. kolo školení práce s~defibrilátorem. Vyjádřil názor, že by bylo pro fakultu výhodné zakoupit cvičný simulátor ? Andulu, s~tím, že by bylo vhodné, aby ve vnitřních předpisech byl zakotven praktický nácvik masáže.

\textit{P. Boreš:} Otevřel diskusi k~nepřítomným, resp. omluveným senátorům.

\textit{P. Habala:} Vyslovil se pro to, aby se omluvy neřešily - buď je senátor přítomen nebo není.

\textit{P. Boreš:} Uvedl, že stačí vzkaz, že senátor nepřijde.

\textit{M. Navara:} Vyjádřil názor, že lidi zvenku by mohlo zajímat, co dělají ti, které zvolili.

\textit{J. Fuka:} Souhlasil s~tím, že by měli senátoři vzkázat, že nepřijdou.
K. Hoffmann: Informoval, že legislativní komise dává dohromady materiály s~ohledem na blížící se přípravu návrhu volebního a jednacího řádu. Otázal se, zda někdo nemá volební a jednací řád platný před rokem 1998.

\textit{Z. Škvor:} Odkázal K. Hoffmanna na sekretariát děkana.

%%%%%%%%%%%%%%%%%%%%%%%%%%%%%%%%%%%%%%%%%%%%%%%%%%%%%%%%%%%%%%%%%%%%%%%%%%%%%%%%
\section{Termín příštího zasedání}
\textit{J. Fuka:} V~souvislosti s~diskusí o~termínu příštího zasedání informoval o~postupu přípravy rozpisu finančních prostředků. Uvedl, že velký senát by měl rozpis schvalovat 25. 4. 2007.

Navrženo řádné zasedání 18. 5. 2007 s~tím, že v~závislosti na průběhu projednávání rozpočtu je možné, že bude svoláno mimořádné zasedání dříve, např. 4. 5. (Bude hledán případný termín v~diskusi na fóru.).

\paragraph{Hlasování o~termínu příštího zasedání:}
\hlasovani{19}{0}{0}
\textbf{Termín příštího zasedání byl schválen. }

\textit{V. Kříha:} Na závěr zasedání poděkoval P. Habalovi za jeho předsednickou práci pro senát.


\paragraph{Seznam termínů:}
\begin{itemize}
\item Termín příštího zasedání: 18. 5. 2007
\item Setkání kandidátů na děkana s~AS FEL: 18. 5. 2007.
\end{itemize}

\paragraph{Seznam usnesení:}
\begin{itemize}
\item AS FEL vysílá studenty K. Bujačka, R. Černocha, M. Půlpitla na semináře ACSA v~Brně (27. 4. a 4. 5. 2007).
\end{itemize}



\bigskip\bigskip\bigskip

V~Praze 26. 4. 2007

\bigskip\bigskip

\begin{center}
\begin{tabular}{p{4cm}p{4cm}p{4cm}}
V. Kříha & P. Boreš & R. Černoch \\
předseda AS FEL & předsedající AS FEL & tajemník AS FEL
\end{tabular}
\end{center}

\end{document}
