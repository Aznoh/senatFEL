\documentclass[a4paper,10pt,notitlepage]{article}

\usepackage[utf8]{inputenc}
\usepackage[czech]{babel}
\usepackage[pdftex]{graphicx}
\usepackage[pdftex]{hyperref}
\usepackage{a4wide}
\usepackage{array}

\makeindex

\title{\textbf{Zápis z~12. řádného zasedání AS FEL}\\konaného dne 7. 3. 2008 od 9.00 hodin}
\author{}\date{} 

% Samostatná časová značka
\newcommand{\ts}[1]{\texttt{[#1]}}
% Časová značka jako začátek odstavce
\newcommand{\tsp}[1]{\noindent \textit{Čas}: \ts{#1}}
% 'Někdo' něco říká
\newcommand{\cl}[1]{\noindent \textbf{#1}:}
% Hlasování
\newcommand{\hl}[3]{\paragraph{Hlasování} \textit{#1}; \textbf{#3} #2.}
% První hlasování obashuje 'pro-proti-zdržel se'
\newcommand{\hlprvni}[3]{\paragraph{Hlasování} \textit{#1}; \textbf{#3}
#2 (pro--proti--zdržel~se).}
% Poznámky zapisovatele
\newcommand{\pozn}[1]{\begin{center}\textit{#1}\end{center}}
% Base-url pro přílohy se nastavuje zde.
\newcommand{\tourl}[1]{http://www.fel.cvut.cz/senat/zapisy/#1}

% kb
\newcommand\uv[1]{\quotedblbase #1\textquotedblleft}%

\begin{document}

\maketitle

\paragraph{Přítomní zaměstnanci:} Petr Boreš, Jindřich Fuka, Petr Habala, Karel Hoffmann, Jan Holub, Josef Kolář, Vítěslav Kříha, Miroslav Lev, Mirko Navara, Jiří Novák, Michal Pěchouček, Petr Skalický

\paragraph{Přítomní studenti:} Karol Bujaček, Mattia Butta, Radomír Černoch, Martin Dobiáš, Jiří Dostál, Martin Půlpitel, Jan Sláma, Monika Žáková

\paragraph{Hosté:} Ivan Jelínek, Pavel Mindl, Zbyněk Škvor, Boris Šimák

%%%%%%%%%%%%%%%%%%%%%%%%%%%%%%%%%%%%%%%%%%%%%%%%%%%%%%%%%%%%%%%%%%%%%%%%%%%%%%
%%%%%%%%%%%%%%%%%%%%%%%%%%%%%%%%%%%%%%%%%%%%%%%%%%%%%%%%%%%%%%%%%%%%%%%%%%%%%%
\section{Schválení programu}

\cl{Navara} Zápis z minulého zasedání nemáme, staženo. Navrhuji také stáhnout bod programu \textit{metodika financování výuky}, komise K7 mě ujistila, že na tom pracuje; budu na průběh dohlížet.

\hlprvni{Schválení programu}{18--0--1}{schváleno}


%%%%%%%%%%%%%%%%%%%%%%%%%%%%%%%%%%%%%%%%%%%%%%%%%%%%%%%%%%%%%%%%%%%%%%%%%%%%%%
%%%%%%%%%%%%%%%%%%%%%%%%%%%%%%%%%%%%%%%%%%%%%%%%%%%%%%%%%%%%%%%%%%%%%%%%%%%%%%
\section{Úprava \uv{Předpisu provádění studijních programů}}

\ts{0:04:30}

\cl{Jelínek} \href{\tourl{zapis12-predpis-stud-delta33-34.doc}}{Článek 33 a 34} byl z části \textit{doktorské studium} přesunut do části \textit{společná ustanovení}, protože se týká všech forem studia. Jednalo se patrně o nedopatření.

\cl{Kříha} Zdůvodnil \href{\tourl{zapis12-predpis-stud-delta36.doc}}{další změnu v \textit{závěrečných ustanoveních}}. Přechodná ustanovení řešila situaci z předchozí verze předpisu. Docházelo ke konfliktu s termíny schvalování předpisu ve velkém senátu a navíc by bez navrhované změny museli studenti dohánět 6 kreditů za publikace.

\hl{Schválení \uv{Předpisu pro provádění studijních předpisů}}{18--0--1}{schváleno}


%%%%%%%%%%%%%%%%%%%%%%%%%%%%%%%%%%%%%%%%%%%%%%%%%%%%%%%%%%%%%%%%%%%%%%%%%%%%%%
%%%%%%%%%%%%%%%%%%%%%%%%%%%%%%%%%%%%%%%%%%%%%%%%%%%%%%%%%%%%%%%%%%%%%%%%%%%%%%
\section{Směrnice děkana pro bakalářské a magisterské zkoušky}

\tsp{0:10:10}

Proděkan Jelínek prezentoval svůj návrh změny studijních předpisů.  Děkan dnes nemá právo přerušit studentovi studium, může být přerušeno jen na žádost studenta. A jsou dobré okamžiky, kdy by to bylo vhodné.

\subsection{Návrh formulací v předpisech}

\tsp{0:12:50}

\cl{Jelínek} Směrnice děkana dnes říká: \uv{Studentovi, kterému byl určen náhradní termín bakalářské obhajoby, může být na jeho žádost přerušeno studium.} Stejná formulace je pro magisterské studium. Návrh zní: \uv{Studentovi, kterému byl určen náhradní termín bakalářské obhajoby, děkan přeruší studium do dalšího termínu bakalářské zkoušky.} Můžeme debatovat o variantě \uv{\dots děkan může přerušit \dots}, také by se mělo počítat s variantou individuálního studijního plánu.

Názor právního oddělení: Násilné přerušení studia je v rozporu s vyššími předpisy. Zákon je v pořádku, ale problém je v předpisu ČVUT, který se naší změně musí přizpůsobit. Navrhovoná změna ve \textit{studijním a zkušebním řádu} by byla: \uv{V případě neúspěšné obhajoby může děkan až do doby obhajoby studium přerušit \dots} Rád bych tuto změnu inicioval.
  
\pozn{Následující průběh zasedání byl kvůli přehlednosti\\do zápisu přestrukturován dle obsahu.}

\subsection{Důvody pro změnu}
\begin{itemize}
\item Student má splněny všechny kredity, ale neodevzdá bakalářskou práci. Zůstává stále studentem a neví, jaké předměty si zapsat. \ts{0:11:00}

\item V doktorském studiu doktorand absolvuje 3 roky, nepožádá o prodloužení studia a čeká na obhajobu. Tehdy je nutná žádost studenta o přerušení. \ts{0:11:30}

\cl{Škvor} Další důvod pro změnu: Děkan nemůže přerušit studium studentovi, který je v bezvědomí -- těžko v takových případech student o přerušení požádá. \ts{0:16:00}

\item Při přerušení nenabíhají studentovi roky studia placené ministerstvem a po znovuzapočetí studia by musel začít platit. \ts{0:21:30}

\item Studenti si zapíšou spoustu předmětů (v prerekvizitách), pak na ně nechodí, neefektivně se využívají prostory.  \ts{0:24:00} \textbf{proti:} Ale peníze na studenta přijdou stejně. Financování se odvíjí od stavu studentů k 31. říjnu a do 30. října dalšího roku se ministerstvo o počet studetů nezajímá. \ts{0:25:00}

\item \cl{Škvor, Jelínek} Nemáme žádný způsob, jak donutit studenty, aby zaplatili školné. Ani jim nemůžeme nezapsat předměty. \ts{0:28:00}

\item \cl{Škvor} Stovkám studentů už přerušení bylo provedeno studijní referentkou a zachránila je tak od ukončení studia. Nedělejme to nadále protiprávně. \ts{0:36:30} 

\end{itemize}


\subsection{Důvody proti změně}
\begin{itemize}
\item Ale student nemusí obhájit ze 2 důvodů: Buď diplomku nestihne, nebo neudělá její obhajobu. V prvním případě nesmíme studentovi studium přerušit -- neměl by přístup do laboratoří, které potřebuje pro její dokončení. \ts{0:17:04} \cl{Jelínek} Proto je tam to slovíčko \uv{může}.
  
\item \cl{Jelínek} Hrozí zneužití ze strany děkana, bylo by dobré to ošetřit. \ts{0:19:00}

\cl{Škvor} Není potřeba bát se děkanových špatných úmyslů. Škvorova hrůzovláda skončila, mějme v děkana důvěru\dots

\item \cl{Půlpitel} Nejsou žádné statistiky o počtu studentů, kterých se problém týká. Kolik je proti, kolika by změna vyhovovala? \ts{0:33:50} \cl{Jelínek} Dobře. Ale jde také o princip. I kdyby to byl 1 student, děkan by na to měl mít právo. \ts{0:35:20}

\end{itemize}



\subsection{Další poznámky}
\begin{itemize}
\item \cl{Jelínek} Bod je mé přání do budoucna, zatím není navrhovaná změna v souladu s vyššími právními předpisy. \ts{0:10:10}

\item \cl{Boreš} Změna může postihnout lepší studenty. Někdo vypadne při předmětové zkoušce~$\rightarrow$~přerušíme mu studium; zato  při neobhájení diplomky bude zvýhodněn. Návrh: Co to udělat tak, že student musí mít ISP nebo přerušit studium a ISP by byl závazný. \ts{0:17:30}

\item \cl{Fuka} Měly by být vyjmenovány důvody, ve kterých může k násilnému přerušení dojít. \ts{0:27:00} Děkan se navíc může o konkrétní důvod opřít. \ts{0:28:50} \textbf{proti:} Nelze dopředu vymyslet všechny případy.

\item \cl{Habala} Jestli-že je shoda na obou stranách, není problém a nic neměňme. Jestliže dojde k nesouladu, musí to znamenat, že pro studenta to výhodné není. Měli bychom předpis nastavit tak, že student může mít obranu -- jedná se totiž o konfliktní ustanovení. Např. \textit{institut odvolání}. Nebo návrh kolegy Boreše, kdy student by si měl vybrat mezi ISP a přerušením. \ts{0:30:33}

\cl{Pěchouček} Souhlasím, jsou případy, kdy student je nečinný a pak by nebyl problém. Právo odvolání by student měl mít. \ts{0:35:30}

\cl{Jelínek} Institut odvolání existuje tak jako tak, proti každému rozhodnutí. \ts{0:39:00}

\cl{Škvor} Odvolání nesmí mít odkladný účinek. Jestliže si student nezapíše předměty do 5 dnů, končí mu studium. Odvolání nemusí vždy stihnout.
\end{itemize}

\pozn{Čas je v závislosti na pozici v textu nadále rostoucí.}

\noindent Proděkan Jelínek žádal o usnesení, které by vyjadřovalo podporu jeho dalším krokům. V diskusi byla navrženo usnesení ve znění: \uv{Nebylo přijato takové usnesení, které by vyjadřovalo, že senát má námitek proti dalším krokům v této věci.} Tato formulace však byla po právu opuštěna. V obecné shodě byl návrh chápán v příliš pracovní podobě (\uv{nástřel}) na to, aby hlasování o jeho podpoře mělo dostatečnou vypovídací hodnotu.

\tsp{0:46:45}

\cl{Kříha} Jde o to, jestli vůbec navrhovat velkému senátu změnu předpisů. Myslím, že v tomto naši podporu návrh má, a můžeme se posunout k dalšímu bodu programu.

\subsection{Jazyk závěrečných prací}

\cl{Skalický} Navrhl změnu předpisu tak, aby závěrečná práce mohla být i ve slovenštině, protože tím porušujeme směrnici EU o menšinových jazycích. A slovenština je menšinovým jazykem v ČR.

\cl{Jelínek} Dobře, do příště to bude.

\cl{Skalický} Je otázka (a na to nemám tak tvrdý názor), zda by řádní studenti českých programů mohli mít závěrečnou práci v češtině nebo slovenštině a jen samoplátci mohli mít práci v angličtině. Donutili bychom studenty, kteří česky vůbec neumí, aby se jazyk naučili.

\cl{Navara} Nic bych nepředepisoval, práce je užitečná, ať je v jakémkoli jazyce.

\cl{Škvor} Návrh, ať každý, kdo bude studovat anglicky, prokáže znalost angličtiny nějakou zkouškou a podobně v češtině pro ty, pro které není čeština národním jazykem.

\tsp{0:53:00}

\cl{Kolář} V jakém jazyce budou prezentace a otázky komise studentům, kteří píšou práci anglicky? To je problém. Jak by student odpovídal?

\cl{Boreš} Hrozí, že se sníží kvalita podobně, jako je to dnes u anglických předmětů, do kterých se cpou studenti, kteří hledají jednodušší průchod studiem.

\subsection{Vyhodnocení anglické ankety}

%---vvv---zde začal editovat JaS---vvv---%
\tsp{0:56:00}


\cl{Jelínek} Jen informace, že anglická anketa byla otevřena tento týden --- to byl požadavek senátu. Běźná anketa se uzavírá příští týden --- požadavek v senátu příští týden uzavřít. A začne rozřazování do oborů.

\cl{Sláma} Kdy byla otevřená anglická verze ankety?

\cl{Jelínek} Před týdnem, nebo toto pondělí.

\cl{Sláma} O dva měsíce později. Před Vánoci bylo avizováno, že bude otevřena 14. ledna.

\cl{Jelínek} Zpoždění bylo způsobeno technickými problémy,\dots
%---^^^---zde přestal editovat JaS---^^^---%



%%%%%%%%%%%%%%%%%%%%%%%%%%%%%%%%%%%%%%%%%%%%%%%%%%%%%%%%%%%%%%%%%%%%%%%%%%%%%%
%%%%%%%%%%%%%%%%%%%%%%%%%%%%%%%%%%%%%%%%%%%%%%%%%%%%%%%%%%%%%%%%%%%%%%%%%%%%%%
\newpage\section{Zprávy z rady vysokých škol}

\tsp{1:05:00}

\cl{Ripka} Připravuje se Bílá kniha, práce na ní se hodně akceleruje. Zároveň je silný politický tlak na vytvoření zcela nového vysokoškolského zákona. Zatím se moc neví, co za tím stojí. Velkým tématem je změna pravidel pro financování vědy a výzkumu.

\begin{itemize}
\item Školné jako utkvělá myšlenka prof. Matějů -- všeřešící záchrana českého školství. Projedává se myšlenka politicky průchodného \textit{odloženého školného}. Můj názor je ten, že se ztrácí motivační efekt, navíc se pak školné přesune do nejtíživější životní etapy studenta, tedy do začátku kariéry. Je třeba navíc zařídit, aby se nesnížil příspěvěk VŠ o vybrané školné.

\item Koncepce řízení VŠ: Existuje tendence omezovat práva volených orgánů. Tedy přesun pravomoci od AS ke správním radám. Někdy je to pochopitelné, ale moje snaha je pravomoci senátů udržet. Můj názor je ten, že by se měla uzákonit odvolatelnost senátů.

\item Kategorizace VŠ, zatím spíše fakult. Můj názor je, že dokud se nenavýší příspěvek školám za studenta (zapomíná se na inflaci apod.), nemá kategorizace smysl. Já budu prosazovat spíše navýšení příspěvku. Navíc je kategorizace už neformálně udělána. Velké VŠ shrábly 90\% za vědu, regionální nemají skoro nic.

\item Návrh na odvolatelnost profesorů, např. při falšování vědeckých údajů. Tohle asi projde.
\end{itemize}

\tsp{1:15:30}

\cl{Pěchouček} Bylo by možné specifikovat tendence k omezení práv senátů?

\cl{Ripka} Tato tendence je směrem k omezení vlivu na finance. Senát by schvaloval koncepci, konkrétní rozpočet by schvalovala správní rada. Na druhou stranu by část správní rady volili zaměstnanci. Dále by rektor měl mít možnost odvolat děkana bez souhlasu akademického senátu.

\cl{Pěchouček} Jak by vypadala správní rada?

\cl{Ripka} Líbivě zní, že by měla být větší role \uv{externích aktérů}. Jenže to je v rozporu s hmotnou odpovědností správní rady.

\cl{Pěchouček} Takže vliv na rozhodování se přesouvá od studentů kam?

\cl{Škvor} Od fakult k rektorům a správním radám s účastí průmyslu a místních zastupitelstev.

\cl{Žáková} Existuje první verze Bílé knihy, kterou mohu poskytnout.

\section{Rozpis financování kateder}

\tsp{1:21:00}

\cl{Mindl} Novou metodiku Kometa máme připravenou k předložení. Anketa není vypuštěna z Komety, ale přidaná byla formulace, že se koeficient ankety může položit $K_A = 1$, pokud by výsledek ankety nebyl relevantní. Vychází se z toho, že sociologické šetření ankety ukázalo její nevhodnost pro použití ve financování.

Metodika je rozšířena o to, že skupina studentů v Sezimově Ústí je považována za jednu paralelku.

\tsp{1:24:50}

\cl{Navara} Anketa má vliv $\pm 2..3\%$, ale jiné vlivy hrají řádově stovky procent. Bylo by vhodné ošetřit je, není potřeba řešit anketu.

\cl{Mindl} Ano, tyto disproporce jsou odůvodnitelné a jsou způsobené kategorizací předmětů a jejich atributy. V současné době chystáme úpravu KOSu tak, aby se tyto atributy do něj natáhly a bylo možné je upravovat v průběhu semestru. Mělo by tím dojít k větší spravedlnosti rozdělování.

Důležitý je vliv klouzavého průměru, kdy finance kateder se průměrují za poslední a současný rok. To způsobuje největší disproporci, bez průměrování vychází cena započitetelné hodiny spravedlivě. Rostoucí katedra chvíli nedostává adekvátní množství peněz, naopak klesající katedru systém chvíli finančně podrží.

\tsp{1:30:50}

\cl{Pěchouček} Neznamená to zamezení vlivu děkana na rozvoj nových, progresivních výzkumných skupin?

\cl{Mindl} Děkan tuto pravomoc má. Není třeba mít strach. Např. do STM stále nabíhají studenti. Tento nárůst se hradí z celé fakulty. Finanční prostředky na KP nabíhají pomalu právě díky klouzavému průměru. To je dobře, protože ostatním katedrám množství peněz klesá pomaleji a katedra má čas udělat potřebná opatření.

Současná metodika tomu pomáhá tím, že zahrnuje \uv{kvalifikovaný odhad} studentů, kteří v budoucnu nastoupí.

Druhá věc je dát děkanovi další pravomoc zásahu do rozpočtu.

\tsp{1:38:00}

\cl{Ripka} Poznámka k \textit{inteligentním budovám:} Nežádáme fakultu o finance, zdroje bereme z externích projektů.

\cl{Škvor} Je ale potřeba nejdřív ty peníze mít. Nějaké peníze musíme ušetřit, při přerozdělování není odkud brát.

\cl{Boreš} Dotaz na časový harmonogram.

\cl{Mindl} Snad to stihneme do 28. 3.

\cl{Fuka} V ekonomické komisi bychom rádi měli data 14 dní dopředu. 

\section{Změna vedení katedry počítačů}

\tsp{1:43:00}

\cl{Děkan} Odkud to mám začít vyprávět?

\cl{Senát} Od začátku.

\cl{Děkan}

\begin{itemize}
\item Do volebního boje jsem šel s představou jednotné fakulty.
\item V červnu mi prof. Tvrdík sdělil představu o spolupráci KP v rámci FEL -- od vytvoření samostatného programu (nic proti) po rozdělení fakulty v celé struktuře včetně senátu (byl jsem proti).
\item V srpnu jsem najmenoval K9 se zadáním vytvořit koncept studijních programů nezávisle na současné struktuře kateder. Byla rozpracována nějaká vize, ale koncepčně se na ní nedalo pracovat dál.
\item S KP jsme se domluvili, že vznikne komise K7 pro ekonomiku, která se měla podívat na financování studijních programů; komise stále pracuje. Mandát má do konce března. 19. prosince jsme se s katedrami dohodli, že ač není ekonomický model, chceme dělat společnou fakultu.
\item Co se stalo? Paralelně se vypracovával návrh na samostatnou fakultu! Děkan o ní vůbec nic nevěděl. Informace o tajně připravovaných programech na KP jsem se dozvěděl náhodou, když jsem na chodbě přátelsky mluvil s prof. Žárou. Teprve před 14 dny jsem dostal přístup na jejich wiki, ve které je vidět, co tvoří.
\end{itemize}


\subsection{Prostory}

\tsp{1:49:30}

\begin{itemize}
\item Říká se, že nové rozdělení prostor na Karlově náměstí vzbudilo velký odpor.
\item Problém je, že jsem neměl podklady. Měl jsem nápad přerozdělit celou fakultu, požádal jsem VIC, aby na předrozdělení udělalo nástroj do informačního systému. Na tom se pracuje.
\item Rozdělení bylo poměřeno množstvím plochy na doktoranda, docenta, \dots
\item Katedra počítačů dostala 20\% navíc, katedra řízení je stále pod normou.
\item Vznikl spor o 40 pracovníků tak, že prof. Tvrdík předložil stav zaměstnanců, ve kterém byli doktorandi uvedeni jako zaměstnanci s normálním úvazkem. Tím vzniklo další řízení.
\item Katedra počítačů tedy nebyla poškozená.
\end{itemize}




\subsection{Další vývoj: Temešvár, Equica}

\tsp{1:53:50}

Požádal jsem, že se jako kolega podívám na zasedání KP na Temešváru. Jel jsem tam neformálně -- pak mi vyčetli, že jsem odvolal Tvrdíka v džínách.

\tsp{1:53:40}

\begin{itemize}
\item Večer před Temešvárem mne navštívil ředitel firmy Equica a předložil mi jakousi smlouvu. Ano, v září jsem dal souhlas s řešením rozvoje KP, ve kterém je prof. Tvrdík zmocněn k věcným jednáním. Od září jsem ale neměl o tomto projektu žádné informace. Smlouva byla podepsána 18. 12. a jsou v ní věci, ke kterým prof. Tvrdík zmocněn nebyl. Já jsem o tom měl vědět! %TODO
\item Prof. Tvrdíka jsem odvolal po návštěvě ředitele Equicy, poté, co jsem viděl, jak se manipuluje se zaměstnanci fakulty. Děkan musí mít partnery, se kterými může spolupracovat.
\item Pak se vzedmula podpisová vlna. Mrzí mne, že mi senátor veřejně předá petici, o které jsem předtím nic nevěděl a neinformuje ani předsedu senátu. (\url{http://www.exfort.org/dopis})

\end{itemize}

\pozn{Děkan začal promítat materiály na projektoru,\\k dispozici jsou k nahlédnutí na děkanátu.}

\begin{itemize}
\item \ts{2:01:00} \textit{Děkan informoval o detailech smlouvy s firmou Equica, kritizoval neoprávněnost prof. Tvrdíka k provedeným krokům a nulovou informovanost děkana.}
\item Nemohl jsem souhlasit s přeprodáváním zaměstnanců. Představte si, že by třeba profesor pracoval jako konzultant firmy XY a dostával za to 50~000? Proč takhle nepřeprodáme i děkana?
\item Když prof. Tvrdík jedná za mými zády, není to partner pro spolupráci, a je to důvodem pro jeho odvolání. Na jeho místo jsem navíc jmenoval opět člena KP -- doc. Šnorka.
\item Jsem zvědavý na další vývoj, protože ve čtvrtek jsem dostal dopis od prof. Žáry, že by chtěl mít zřízené samostatné pracoviště.
\item \ts{2:12:00} \textit{Děkan přečetl \href{\tourl{zapis12-kukas-program.doc}}{program výjezdního zasedání KP} (tzv. KUKAS)}.
\end{itemize}


\subsection{Páteční schůzka vedení s KP a diskuse v Zengerově posluchárně}

\tsp{2:14:20}

Prof. Tvrdík mě informoval, že jsem jej odvolal neoprávněně; patrně po konzultaci s právničkou ČVUT. Paní doktorkou Lindnerovou jsem byl odkázán na jinou právní kancelář. Mimochodem jsem se dozvěděl, že již 14 dní před tím se prof. Tvrdík ptal na možné odvolání. Obrátil jsem se na právníky ministerstva školství a ti potvrdili moje rozhodnutí.

Před senátem říkám, že jsem proti prof. Tvrdíkovi jako vedoucímu katedry počítačů. Nemám nic proti jeho působení na FEL, nemám nic proti jeho profesuře a byl bych rád, kdyby nadále vědecky pracoval na naší fakultě.

\tsp{2:21:00}

\cl{Kolář} \textit{Informoval o průběhu schůzky:} Na schůzce začal pan rektor mluvit o důvodech setkání -- vyjasnění statutu quo. Jde o to, jaký tým bude jednat o vytvoření nové fakulty a jaký bude jeho vztah k FEL.

Já bych chtěl zdůraznit, že vztahy na KP se drží na dobré úrovni i přes jmenování nového vedoucího. Při dalším jednání bude hlavní, aby nevznikl tlak na výuku studentů. Nechceme si brát studenty jako rukojmí.

Pak proběhla diskuse o tom, co se vlastně má konat. Z toho vyplývá diskuse o tom, jestli budeme \textit{dělit}, nebo \textit{vytvářet novou fakultu}. Na tom závisí osud studijních programů. Rektor říkal, že může proces označit jako rozdělení. Děkan nesouhlasí, protože jeho úkolem není budovat fakultu informatiky, ale rozvíjet fakultu elektrotechnickou. Ale vytvoření nové fakulty nebude bránit.

\medskip

\cl{Šimák} Já musím dělat nové studijní programy a zajistit ty současné, jako je STM. Jak mohu dát STM jako základ nové fakulty? Nová fakulta musí sama podat akreditace. Navíc na STM se podílí více kateder.

Navíc má naše fakulta svoje vlastní pojetí informatiky.

\subsection{Diskuse}

\subsubsection{Studijní programy, STM}

\cl{Pěchouček} Já už třetí volební období horuji za silný informatický program, čistou informatiku. Ale způsob, jak KP v této oblasti jedná je velmi nátlakový. Nebyla totiž nalezena podpora natož konsensus ze strany AO. STM se prosadilo nátlakem skrz senát. KP z toho nátlaku viním; je ale třeba vzít v úvahu, že vyhrocená situace nevznikla sama od sebe.  Zbytek fakulty totiž dlouho neuznával čistou informatiku. To vidím jako důvod, proč situace dospěla do dnešního stavu, a proto nesouhlasím, že by bylo dobré fakultu rozdělit proto, aby zde zůstala \uv{čistá elektrotechnika}. Síla FEL je právě ve spojení elektrotechniky a informatiky.
  
\cl{Děkan} Já jsem pro, aby K9 vytvořila samostatný informatický program. A to nezávisle na ostatních katedrách, které by byly potencionálně proti.

\cl{Pěchouček} To je myslím důkazem, že nejsme ochotni se dohodnout. V K9 byli 2 zástupci KP a přesto se v zápisu objevil dlouhý text kurzívou, který se vyhraňuje proti celému zbytku textu.

\cl{Kolář} S ostatními kolegy z různých fakult jsme se chtěli dohodnout na pokračování STM. Domluvili jsme se, že není dobré hnát STMáky do magisterského programu E+I. Nebyla to pirátština vůči elektrofakultě. Je v zájmu všech, aby lidé z STM mohli navazovat. Proč to ještě nebylo zpracované?

\tsp{2:47:30}

\cl{Děkan} Věc jsem chtěl řešit koncepčně, říkal jsem to od začátku, najmenoval jsem proto K9, celé je to transparentní.

\cl{Skalický} Já nevidím ve studentech STM problém. Díky Boloňské deklaraci mohou nastoupit do libovolného studijního programu. Nevytvářejme problém tam, kde není.

\cl{Pěchouček} Já se bojím, že právě toto je argument, který fakultu rozděluje.

\tsp{3:45:50} \textit{Zbytek bodu je přesunut na toto místo z úplného závěru diskuse.}

\textbf{Škvor} ze závěru diskuse: Budu parafrázovat. Zaslechl jsem z KP, že \uv{zbytek FELu si dělá nárok na náš program STM.} Také rozumím tomu, že zbytek fakulty považuje program za svůj -- spoustu financí do STM investovali tím, že měli menší podíl financí ze společného rozpočtu. Proto si nemyslím, že je dobré, aby STM přešlo na novou fakultu informatiky.

\cl{Kolář} Nárůst STM znamenal jen kompenzaci poklesu z E+I. Navíc STM vydělává na 1 kontaktní hodinu mnohem více.

\subsubsection{Equica}

\tsp{2:51:00}

\cl{Kolář} Nemám přesné informace o vývoji jednání, nebyl jsem ve vedení KP. A uznávám, že prof. Tvrdík asi udělal formální chyby. Existují příklady jiných kateder, které spolupracují s průmyslem a přináší tak finance do svého rozpočtu. A když to udělá KP, je to problém? Myslíte si, že $50 000$ jsou neadekvátní peníze zaměstnancům? Spolupráce je běžná a ostře nesouhlasím s označením \uv{vyprodávání zaměstnanců}.

\subsubsection{Přidělování prostor}

\tsp{2:56:00}

\noindent \textit{Proběhla krátká výměna názorů k průběhu vyjednávnání o přidělování prostor na Karlově náměstí, po které proděkan Mindl představil svoji prezentaci.}

\cl{Mindl} Rozdělení proběhlo pro \textit{pracovny} (ne výukové místnosti). Metodika, kterou jsem použil, říká, kolik prostor má být na docenta, doktoranda apod.

\tsp{3:05:00}

\noindent \textit{Proděkan představil úpravy přidělených ploch dle materiálů v příloze:}
\begin{itemize}
\item \href{\tourl{zapis12-narokovane_plochy.xls}}{Orientační tabulka pro výpočet nárokových ploch pracoven na Karlově náměstí - vyhodnocení stavu po dislokačním příkazu PD 2/2008}
\item \href{\tourl{zapis12-plochy_dle_rozdeleni.xls}}{Plochy na KN po příkazu děkana 2/2008 s plochami pracoven v půdní vestavbě}
\end{itemize}

\cl{Mindl} Do užitné plochy místnosti se počítá vše s výškou stropu nad 120~cm. Vycházelo by to pro půdní vestavbu příliš tvrdě, a proto jsem požádal architekta, aby se toto kritérium posunulo na 130~cm.

Podle projektu mělo pro KP přibýt 66 židlí, Při kolaudaci se 3 místnosti označily jako technický prostor ($\rightarrow$ 60 židlí) a kvůli zkosenému stropu a vytvoření konferenční místnosti jsme se dohodli na 56 místech.

V rámci druhého patra se ustoupilo ještě o 1 místnost směrem ke katedře kybernetiky. KP chtěla další pracovny ve 3. patře. Pro prof. Tvrdíka to nebylo ani tak přijatelné, protože by návrh katedru rozdrobil.

\cl{Černoch} Kolik pracovních míst navíc to pro katedru počítačů znamená?

\cl{Mindl} Z hlediska židlí je to finálně +1 židle.

\medskip

\tsp{3:22:30}

\cl{Kolář} Není možné separovat prostory pro pro sezení lidí od ostatních prostor katedry. Existuje notoricky známý \href{\tourl{zapis12-vyuziti_laboratori.png}}{diagram využití laboratoří na Karlově náměstí}. Přesně jak říkal proděkan Mindl, vedoucí může rozhodnout, jak kterou místnost využije. Ve skutečnosti to tak bylo za prof. Kučery, za děkana Šimáka je tomu jinak. Toto je jasný proti-informatický signál. Tímto návrhem bylo zamítnuto vytvoření speciálních laboratoří, ve kterých mohou pracovat studenti na závěrečných pracích.

\cl{Pěchouček} V tom, co říkáte, je velká míra racionality. Jenže jde o to, jak zohlednit, jestli místnost je dlouhá nudle, nebo má rozumný tvar.

\cl{Děkan} Já si myslím, že ploch budeme mít zanedlouho tolik, že nebudeme vědět, co s nimi. Viz Kladno, nové budovy architektů, nadstavba nad Studentským domem atd., atd.


\subsubsection{Vytvoření nového pracoviště počítačové grafiky}

\tsp{3:33:00}

\cl{Boreš} Prof. Žára požádal děkana o vytvoření nové katedry. Chápu to dobře, že o tom KP nic neví?

\cl{Kolář} Minulé ani současné vedení katedry počítačů se nevyjádřilo o dopise, ve kterém prof. Žára žádost zasílá. Ale nemělo by se uvažovat o odloupnutí tu grafiky z KP, tu kousku jinde\dots Žádost prof. Žáry je navíc obdobou aktivit prof. Tvrdíka, jen o úroveň níže.

\cl{Navara} Já bych tu viděl rozdíl. Myslím si, že prof. Tvrdík nebere věci za správný konec. V případě fakulty informatiky jsem první dokument dostal před 20 minutami.

\tsp{3:38:00} \textit{Děkan shrnul svoje postoje a navrhl společnou prohlídku prostor na Karlově náměstí.}



\section{Příští zasedání}

\hl{Příští termín zasedání AS FEL 4. 4.}{15--1--1}{schváleno}
\hl{Přes-příští termín zasedání AS FEL 25. 4.}{16--1--1}{schváleno}
\hl{Přes-přes-příští termín zasedání AS FEL 23. 5.}{15--1--2}{schváleno}

\medskip

\tsp{3:59:30}

Závěrem prof. Hoffmann informoval o návrhu nového \textit{Volebního a jednacího řádu}. Prof. Navara požádal o průběžné informování o práci na tomto dokumentu, s čímž prof. Hoffmann souhlasil.



\bigskip\bigskip\bigskip\bigskip\bigskip\bigskip\bigskip\bigskip\bigskip

V~Praze 24. 3. 2008

\bigskip\bigskip

\begin{center}
\begin{tabular}{p{4cm}p{4cm}p{4cm}}
V. Kříha & M. Žáková & R. Černoch \\
předseda AS FEL & předsedající AS FEL & tajemník AS FEL
\end{tabular}
\end{center}


\end{document}
